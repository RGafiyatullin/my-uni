\documentclass[12pt]{article}
\usepackage{amsmath}
\usepackage[utf8]{inputenc}
\usepackage[russian]{babel}
\usepackage{color}
\usepackage[usenames,dvipsnames]{xcolor}
\usepackage{graphicx}

\title{Геометрия и алгебра. Контрольная работа №2 - Роман Гафиятуллин (192001-04)}
\author{Роман Гафиятуллин\\ БГУИР}
\begin{document}
	\begin{titlepage}
		\begin{center}
			{\Large Геометрия и алгебра. \\ Контрольная работа №2 \\ Роман Гафиятуллин (192001-04)}
		\end{center}
	\end{titlepage}
	\clearpage
	\paragraph{04-2.2} Исследовать на линейную зависимость над R систему векторов. Векторы указаны ниже в соответствии с вариантом.

	\ensuremath{
		\\
		v_{1} = 2 \\
		v_{2} = sin x \\
		v_{3} = sin ^2 x \\
		v_{4} = cos ^2 x \\
		\\
		\alpha_{1} v_{1} + \alpha_{2} v_{2} + \alpha_{3} v_{3} + \alpha_{4} v_{4} = 0 \\
		\alpha_{1} \cdot 2 + \alpha_{2} sin x + \alpha_{3} sin ^2 x + \alpha_{4} cox ^2 x = 0 \\
		\\
	}
	\ensuremath{
		2 \alpha_{1} + \alpha_{2} sin x + \alpha_{3} sin ^2 x + \alpha_{4} cos ^2 x = 0 \\
	}
	\\
	Дифференцируем по \ensuremath{x} \\
	\ensuremath{
		0 + \alpha_{2} cos x + 2\alpha_{3} sin x \cdot cos x + 2 \alpha_{4} cos x (-sin x) = 0 \\
	}
	\ensuremath{
		\\
		\alpha_{1} v_{1} + \alpha_{2} v_{2} + \alpha_{3} v_{3} + \alpha_{4} v_{4} = \alpha_{2} cos x + 2\alpha_{3} sin x \cdot cos x - 2 \alpha_{4} cos x \cdot sin x
		\\
		\\
		\left\{\begin{matrix}
		x = 0 ^{\circ} \\
		2 \alpha_{1} + \alpha_{4} = \alpha_{2}
		\end{matrix}\right.
		\\
		\left\{\begin{matrix}
		x = 90 ^{\circ} \\
		2 \alpha_{1} + \alpha_{2} + \alpha_{3} = 0
		\end{matrix}\right.
	}
	\\
	\\
	Векторы линейно зависимы

	%%%%%%%%%%%%%%%%%%%%%%%%%%%%%%%%%%%%%%%%%%%%%%%%%%%%%%%%%%%%%%%%%%%%%%%%%%%%%%%%%%%%%%%%%%%%%%%%%%%%%%%%%%%%%%%

	\clearpage
	\paragraph{04-2.3} Найти координаты вектора \ensuremath{x} в базисе \ensuremath{\{\vec{u_{1}}, \vec{u_{2}}, \vec{u_{3}}\}}, если известны его координаты в базисе \ensuremath{\{\vec{e_{1}}, \vec{e_{2}}, \vec{e_{3}}\}}. Разложения векторов \ensuremath{ \vec{u_{1}}, \vec{u_{2}}, \vec{u_{3}} } по базису \ensuremath{\{\vec{e_{1}}, \vec{e_{2}}, \vec{e_{3}}\}} и координаты вектора \ensuremath{x} в этом базисе даны ниже в соответствии с вариантом.

	\ensuremath{
		\\
		\left\{\begin{matrix}
			\vec{u_{1}} = \vec{e_{1}} + \vec{e_{2}} + \frac{3}{2}\vec{e_{3}} \\
			\vec{u_{2}} = 3 \vec{e_{1}} - 1 \vec{e_{2}} \\
			\vec{u_{3}} = - \vec{e_{1}} + \vec{e_{2}} + \vec{e_{3}} \\
		\end{matrix}\right.
		\\\\
		\vec{x} = 2 \vec{e_{1}} + 4 \vec{e_{2}} + 1 \vec{e_{3}}
	}
	\\\\
	Матрица перехода
	\ensuremath{
		\\
		\begin{bmatrix}
			1 &	3 &	-1 \\
			1 &	1 &	1 \\
			\frac{3}{2} & 0 & 1
		\end{bmatrix}
	}
	\\\\
	Обратная ей матрица
	\ensuremath{
		\\
		\begin{bmatrix}
			\frac{1}{4} & -\frac{3}{4} & 1 \\
			\frac{1}{8} & \frac{5}{8} & -\frac{1}{2} \\
			-\frac{3}{8} & \frac{9}{8} & -\frac{1}{2}
		\end{bmatrix}
	}
	\\\\
	Преобразуем вектор \ensuremath{\vec{x}}
	\ensuremath{
		\\
		\begin{bmatrix}
			\frac{1}{4} & -\frac{3}{4} & 1 \\
			\frac{1}{8} & \frac{5}{8} & -\frac{1}{2} \\
			-\frac{3}{8} & \frac{9}{8} & -\frac{1}{2}
		\end{bmatrix}
		\cdot
		\begin{bmatrix}
			2 \\ 4 \\ 1
		\end{bmatrix}
		=
		\begin{bmatrix}
			-\frac{3}{2} \\ \frac{9}{4} \\ \frac{13}{4}
		\end{bmatrix}
	}
	\\\\
	Ответ: 
	\ensuremath{\vec{x}} в базисе \ensuremath{\{\vec{u_{1}}, \vec{u_{2}}, \vec{u_{3}}\}} --- \ensuremath{\begin{bmatrix}
		-\frac{3}{2} & \frac{9}{4} & \frac{13}{4}
	\end{bmatrix}}

	%%%%%%%%%%%%%%%%%%%%%%%%%%%%%%%%%%%%%%%%%%%%%%%%%%%%%%%%%%%%%%%%%%%%%%%%%%%%%%%%%%%%%%%%%%%%%%%%%%%%%%%%%%%%%%%

	\clearpage
	\paragraph{04-2.4} Определить размерность над \ensuremath{R} и найти какой-нибудь базис линейного пространства решений однородной системы линейных уравнений. Указать общее и частное решения системы. Системы уравнений даны ниже в соответствии с вариантами.

	\ensuremath{
		\\
		\left\{\begin{matrix}
			6 x_{1} - 9 x_{2} + 21 x_{3} - 3 x_{4} - 12 x_{5} = 0 \\
			-4 x_{1} + 6 x_{2} - 14 x_{3} + 2 x_{4} + 8 x_{5} = 0 \\
			2 x_{1} - 3 x_{2} + 7 x_{3} - x_{4} - 4 x_{5} = 0 \\
		\end{matrix}\right.
		\\
		\begin{bmatrix}
			6 & -9 & 21 & -3 & -12 \\
			-4 & 6 & -14 & 2 & 8 \\
			2 & -3 & 7 & -1 & -4 \\
		\end{bmatrix}
		\sim 
		\begin{bmatrix}
			1 & -\frac{3}{2} & \frac{7}{2} & -\frac{1}{2} & -2 \\
			0 & 0 & 0 & 0 & 0 \\
			0 & 0 & 0 & 0 & 0 \\
		\end{bmatrix}
		\\
		rank A = 1 \\
		dim V = n - rank A = 5 - 1 = 4 \\
		\\
		x_{1} = \frac{3}{2} x_{2} - \frac{7}{2}x_{3} + \frac{1}{2} x_{4} + 2 x_{5} \\
		\\
		X =\begin{pmatrix}
			\frac{3}{2} x_{2} - \frac{7}{2}x_{3} + \frac{1}{2} x_{4} + 2 x_{5} \\
			x_{2} \\
			x_{3} \\
			x_{4} \\
			x_{5} \\
		\end{pmatrix}
		= \\
		\begin{pmatrix}
			\frac{3}{2} x_{2} \\
			x_{2} \\
			 \\
			 \\
			 \\
		\end{pmatrix}
		+
		\begin{pmatrix}
			- \frac{7}{2}x_{3} \\
			\\
			x_{3} \\
			\\
			\\
		\end{pmatrix}
		+
		\begin{pmatrix}
			\frac{1}{2} x_{4} \\
			\\
			\\
			x_{4} \\
			\\
		\end{pmatrix}
		+
		\begin{pmatrix}
			2 x_{5} \\
			\\
			\\
			\\
			x_{5} \\
		\end{pmatrix} = \\
		x_{2} \begin{pmatrix}
			\frac{3}{2} \\
			1 \\
			\\
			\\
			\\
		\end{pmatrix}
		+
		x_{3} \begin{pmatrix}
			-\frac{7}{2} \\
			\\
			1 \\
			\\
			\\
		\end{pmatrix}
		+
		x_{4} \begin{pmatrix}
			\frac{1}{2} \\
			\\
			\\
			1 \\
			\\
		\end{pmatrix}
		+ x_{5} \begin{pmatrix}
			2 \\
			\\
			\\
			\\
			1 \\
		\end{pmatrix}
	}
	\\\\
	Ответ:
	\ensuremath{
		\\
		dim V = 4
		\\
		V_{1} = \begin{pmatrix}
			\frac{3}{2} \\
			1 \\
			\\
			\\
			\\
		\end{pmatrix}
		;
		V_{2} = \begin{pmatrix}
			-\frac{7}{2} \\
			\\
			1 \\
			\\
			\\
		\end{pmatrix}
		;
		V_{3} = \begin{pmatrix}
			\frac{1}{2} \\
			\\
			\\
			1\\
			\\
		\end{pmatrix}
		;
		V_{4} = \begin{pmatrix}
			2 \\
			\\
			\\
			\\
			1 \\
		\end{pmatrix}
	}	

	%%%%%%%%%%%%%%%%%%%%%%%%%%%%%%%%%%%%%%%%%%%%%%%%%%%%%%%%%%%%%%%%%%%%%%%%%%%%%%%%%%%%%%%%%%%%%%%%%%%%%%%%%%%%%%%

	\clearpage
	\paragraph{04-2.5} Пусть \ensuremath{\vec{x} = \begin{bmatrix} x_{1} & x_{2} & x_{3} \end{bmatrix}}. Является ли линейными преобразование \ensuremath{A}? Координаты \ensuremath{A x} даны ниже в соответствии с вариантом.

	\ensuremath{
		A \vec{v} = \begin{bmatrix}
			3 v_{1} + 2 v_{2} + v_{3} \\
			v_{3} \\
			2 v_{1} - 3 v_{2} - 4 v_{3}
		\end{bmatrix}
		\\
		\vec{x} = \begin{bmatrix} x_{1} & x_{2} & x_{3} \end{bmatrix} \\
		\vec{y} = \begin{bmatrix} y_{1} & y_{2} & y_{3} \end{bmatrix} \\
		\\\\
		A[ \alpha \vec{x} + \beta \vec{y} ] = \begin{bmatrix}
			\alpha x_{3} + 3 (\alpha x_{1} + \beta y_{1}) + 2 (\alpha x_{2} + \beta y_{2}) + \beta y_{3} \\
			\alpha x_{3} + \beta y_{3} \\
			2 ( \alpha x_{1} + \beta y_{1}) - 3 ( \alpha x_{2} + \beta y_{2} ) - 4 ( \alpha x_{3} + \beta y_{3} )
		\end{bmatrix} \\\\
		\alpha A \vec{x} + \beta A \vec{y} = \begin{bmatrix}
			\alpha ( 3 x_{1} + 2 x_{2} + x_{3} ) + \beta ( 3 y_{1} + 2 y_{2} + y_{3} ) \\
			\alpha x_{3} + \beta y_{3} \\
			\alpha ( 2 x_{1} - 3 x_{2} - 4 x_{3} ) + \beta ( 2 y_{1} - 3 y_{2} - 4 y_{3} ) \\
		\end{bmatrix} \\\\
		A[ \alpha \vec{x} + \beta \vec{y} ] = \alpha A \vec{x} + \beta A \vec{y}
	}
	\\\\
	Ответ: преобразование является линейным.


	%%%%%%%%%%%%%%%%%%%%%%%%%%%%%%%%%%%%%%%%%%%%%%%%%%%%%%%%%%%%%%%%%%%%%%%%%%%%%%%%%%%%%%%%%%%%%%%%%%%%%%%%%%%%%%%

	\clearpage
	\paragraph{04-2.6} Доказать линейность, найти матрицу в базисе {i, j, k}, область значений, ядро, ранг и дефект оператора. Преобразования трехмерного векторного про-странства над R указаны ниже в соответствии с вариантами.
	\\\\
	Оператор: зеркальное отражение относительно плоскости OYZ. \\
	\ensuremath{
		A = \begin{bmatrix}
			-1 & 0 & 0 \\
			0  & 1 & 0 \\
			0  & 0 & 1
		\end{bmatrix}
		\\\\
		\vec{x} = \begin{bmatrix} x_{1} & x_{2} & x_{3} \end{bmatrix} \\\\
		\vec{y} = \begin{bmatrix} y_{1} & y_{2} & y_{3} \end{bmatrix} \\\\
		\alpha A \vec{x} + \beta A \vec{y}
		= 
		\begin{bmatrix} 
			-\alpha x_{1} &
			\alpha x_{2} &
			\alpha x_{3} 
		\end{bmatrix}
		+
		\begin{bmatrix} 
			-\beta y_{1} &
			\beta y_{2} &
			\beta y_{3}
		\end{bmatrix}
		= \\\\ =
		\begin{bmatrix}
			-\alpha x_{1} - \beta y_{1} &
			\alpha x_{2} + \beta y_{2} &
			\alpha x_{3} + \beta y_{3} &
		\end{bmatrix} = A( \alpha \vec{x} + \beta \vec{y} )
	} \\\\\\
	Ответ:
	Преобразование линейно. \\\\
	Cтолбцы \ensuremath{A} линейно независимы, следовательно: \\\\
	\ensuremath{
		rank A = 3;
		im A = R ^3 ; dim (im A) = 3 \\
		ker A = {} ; dim (ker A) = 0
 	}



 	%%%%%%%%%%%%%%%%%%%%%%%%%%%%%%%%%%%%%%%%%%%%%%%%%%%%%%%%%%%%%%%%%%%%%%%%%%%%%%%%%%%%%%%%%%%%%%%%%%%%%%%%%%%%%%%

	\clearpage
	\paragraph{04-2.9} Найти собственные значения и собственные векторы матрицы A. Матрицы A указаны в табл. 5 в соответствии с вариантами. \\\\
	\ensuremath {
		M = \begin{bmatrix}
			5 & -1 & -1 \\
			0 &  4 & -1 \\
			0 & -1 &  4
		\end{bmatrix} \\
	} \\ \\
	Характеристический многочлен: \\
	\ensuremath{
		x ^3 - 13 x ^2 + 55 x - 75
	} \\ \\
	Действительные собственные значения: \\
	\ensuremath{
		\begin{bmatrix}
			3 & 5 & 5
		\end{bmatrix}
	} \\ \\
	Собственные векторы: \\
	\ensuremath{
		\vec{e}_{1(3)} = \begin{bmatrix} 1 & 1 & 1 \end{bmatrix} \\
		\vec{e}_{2(5)} = \begin{bmatrix} 1 & 0 & 0 \end{bmatrix} \\
		\vec{e}_{3(5)} = \begin{bmatrix} 0 & -1 & 1 \end{bmatrix} \\
	}

\end{document}

