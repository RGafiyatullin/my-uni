\documentclass[12pt]{article}
\usepackage{amsmath}
\usepackage[utf8]{inputenc}
\usepackage[russian]{babel}
\usepackage{color}
\usepackage[usenames,dvipsnames]{xcolor}
\usepackage{graphicx}

\title{Математический анализ. Контрольная работа №1 - Роман Гафиятуллин (192001-04)}
\author{Роман Гафиятуллин\\ БГУИР}
\begin{document}
	\begin{titlepage}
		\begin{center}
			{\Large Математический анализ. \\ Контрольная работа №1 \\ Роман Гафиятуллин (192001-04)}
		\end{center}
	\end{titlepage}
	%%%%%%%%%%%%%%%%%%%%%%%%%%%%%%%%%%%%%%%%%%%%%%%%%%%%%%%%%%%%%%%%%%%%%%%%%%%%%%%%%%%%%%%%%%%%%%%%%%%%%%%%%%%%%%%%%
	\clearpage
	\paragraph{04-1.1} Найти пределы функций, не пользуясь правилом Лопиталя. \\
	\begin{description}
		\item[а)]
			\ensuremath{
				\lim_{x \to \infty} \frac{3 x ^2 - 5 x + 4}{x ^3 - x + 1} = \\
				= \lim_{x \to \infty} 
					\frac{x ^2}{x ^2} \cdot 
					\frac
						{ 3 - \frac{5}{x} + \frac{4}{x ^2} }
						{ x - \frac{1}{x} + \frac{1}{x ^2} }
				= \\
				= lim_{x \to \infty} 
					\frac
						{ 3 - 0 + 0 }
						{ x - 0 + 0 }
				= \\
				= lim{x \to \infty}
					\frac{3}{x} = {\bf 0}
			}
		\item[б)]
			\ensuremath{
				\lim_{x \to 3} \frac{2 x ^2 - 9 x + 9}{x ^2 - 5 x + 6} = \\
				= \lim_{x \to 3}
					\frac{x ^2}{x ^2} \cdot 
					\frac
						{2 - \frac{9}{x} + \frac{9}{x ^2} }
						{1 - \frac{5}{x} + \frac{6}{x ^2} } = \\
				= \lim_{x \to 3}
					\frac
						{ 2 - 3 + 1 }
						{ 1 - \frac{5}{3} + \frac{6}{9} } = \\
				= \lim_{x \to 3}
					\frac
						{ 2 - 3 + 1 }
						{ 1 - \frac{7}{3} } = {\bf 0}
			}
		\item[в)]
			\ensuremath{
				\lim_{x \to 1} \frac{\sqrt{5 - x} - \sqrt{3 + x}}{x - x ^2} = ? \\
			}
		\item[г)]
			\ensuremath{
				\lim_{x \to 0} \frac{3 x \cdot tg(x)}{sin ^2 (x)} = \\
				= \lim_{x \to 0} 
					\frac
						{ \frac{3 x \cdot sin x}{cos x}}
						{ sin ^2 x } = \\
				\textcolor{Cyan}{// \lim_{x \to 0} sin x \sim x } \\
				= \lim_{x \to 0}
					\frac
						{ \frac{ 3 x ^2 }{cos x} }
						{ x ^2 } = \\
				= \lim_{x \to 0}
					\frac
						{ 3 x ^2 }
						{ x ^2 \cdot cos x } = \frac{3}{1} = {\bf 3}
			}
		\item[д)]
			\ensuremath{
				\lim_{x \to \infty} (\frac{2 x + 5}{2 x - 1}) ^{3 - x} = ?\\
			}
	\end{description}

	%%%%%%%%%%%%%%%%%%%%%%%%%%%%%%%%%%%%%%%%%%%%%%%%%%%%%%%%%%%%%%%%%%%%%%%%%%%%%%%%%%%%%%%%%%%%%%%%%%%%%%%%%%%%%%%%%
	\paragraph{04-1.2} Даны комплексные числа. Необходимо: а) выполнить действия в алгебраической форме; б) найти тригонометрическую форму числа \ensuremath{z} и вычислить \ensuremath{z ^{20}} ; найти корни уравнения \ensuremath{w ^3 + z = 0} и отметить их на комплексной плоскости. \\

	\begin{description}
		\item[а)]
			\ensuremath{
				( \frac
					{ 3 - i }
					{ -2 - 6i }
				) ^3 
				= \frac	
					{18 - 26i}
					{208 - 144i}
				= {\bf -\frac{i}{8} }
			}
		\item[б)]
			\ensuremath{
				z = 1 + \sqrt{3}i \\
				\\
				r_{z} = \sqrt{Re(z) ^2 + Im(z) ^2} = 2 \\
				\phi_{z} = arctg( \frac{ Im(z) }{ Re(z) } ) = \frac{\pi}{3} \\
				\\
				r_{z ^{20}} = r_{z}^{20} = 1048576 \\
				\phi_{z ^{20}} = \phi_{z} \cdot 20 \ mod \ 2\pi = \frac{2 \pi}{3} \\
				Re(z ^{20}) = r_{z ^{20}} \cdot cos(\phi_{z ^{20}}) = -524288\\
				Im(z ^{20}) = r_{z ^{20}} \cdot i \cdot sin(\phi_{z ^{20}}) = 524288 \, i \sqrt{3}\\
			}
	\end{description}

	%%%%%%%%%%%%%%%%%%%%%%%%%%%%%%%%%%%%%%%%%%%%%%%%%%%%%%%%%%%%%%%%%%%%%%%%%%%%%%%%%%%%%%%%%%%%%%%%%%%%%%%%%%%%%%%%%
	\paragraph{04-1.3} Исходя из определения равенства множеств и операций над множествами, доказать тождество и проверить его с помощью диаграммы Эйлера – Венна. \\

	\ensuremath{
		A \cap ( B \cup C ) = (A \cap B) \cup (A \cap C)
		\forall a 
	}

	%%%%%%%%%%%%%%%%%%%%%%%%%%%%%%%%%%%%%%%%%%%%%%%%%%%%%%%%%%%%%%%%%%%%%%%%%%%%%%%%%%%%%%%%%%%%%%%%%%%%%%%%%%%%%%%%%
	\paragraph{04-1.4}\begin{description}
		\item[а)] в разложении \ensuremath{(x ^k + y ^P) ^n} найти члены, содержащие \ensuremath{ x ^\alpha }, \\
			если
			\ensuremath{k = 2, p = 1, n = 10, \alpha = 16}
		\item[б)] в разложении \ensuremath{(x + y + z + w) ^m} найти члены, содержащие \ensuremath{ x ^\gamma }, \\
			если
			\ensuremath{m = 9, \gamma = 6}
	\end{description}



\end{document}

