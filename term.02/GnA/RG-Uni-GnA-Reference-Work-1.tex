\documentclass[12pt]{article}
\usepackage{amsmath}
\usepackage[utf8]{inputenc}
\usepackage[russian]{babel}
\usepackage{color}
\usepackage[usenames,dvipsnames]{xcolor}
\usepackage{graphicx}
\usepackage{amsfonts}


\title{Геометрия и алгебра (часть 2). Контрольная работа №1 - Роман Гафиятуллин (192001-04)}
\author{Роман Гафиятуллин\\ БГУИР}
\begin{document}
	\begin{titlepage}
		\begin{center}
			{\Large Геометрия и алгебра (часть 2). \\ Контрольная работа №1 \\ Роман Гафиятуллин (192001-04)}
		\end{center}
	\end{titlepage}
	\clearpage
	\paragraph{04-1.1} Найти \ensuremath{gcd(a, b)} и записать соотношение Безу, используя расширенный алгоритм Евклида или его обратную прогонку.
	\\
	\ensuremath{
		a = -735 \\
		b = 225 \\
		\\
		\\
		\begin{matrix}
			\text{Dividend} & \text{Divisor} & \text{Quotient} & \text{Remainder} \\
			\hline
			225	&	-735	&	q_{1} = 0 	&  r_{1} = 225 \\
			\hline
			-735 &  225		&   q_{2} = -4  &  r_{2} = 165 \\
			\hline
			225 &   165     &   q_{3} = 1  &  r_{3} = 60 \\
			\hline
			165 &   60      &   q_{4} = 2  &  r_{4} = 45  \\
			\hline
			60 &    45     &   q_{5} = 1  & r_{5} = 15 \\
			\hline
			45 &    15     &   3  & 0
		\end{matrix}
		\\
		\\
		gcd(-735, 225) = -735 x + 225 y
		\\
		r_{1} = b \\
		r_{2} = a - ( q_{2} \cdot r_{1} ) = a - (- 4) b = a + 4 b \\
		r_{3} = r_{1} - q_{3} \cdot r_{2} = b - ( a + 4 b ) = - ( a + 3b ) \\
		r_{4} = r_{2} - q_{4} \cdot r_{3} = (a + 4 b) - 2 ( - ( a + 3b ) ) = a + 4b + 2a + 6b = 3a + 10b \\
		r_{5} = r_{3} - q_{5} \cdot r_{4} = - ( a + 3b ) - (3a + 10b) = -4a - 13b = gcd(a, b) \\
		\\
		\text{B\'{e}zout's identity: } gcd(a, b) = -4a - 13b
	}
	\clearpage
	%%%%%%%%%%%%%%%%%%%%%%%%%%%%%%%%%%%%%%%%%%%%%%%%%%%%%%%%%%%%%%%%%%%%%%%%%%%%%%%%%%%%%%%%%%%%%%%%%%%%%%%%%%%%%%
	\paragraph{04-1.2} Найти gcd(a, b) и lcm(a, b) используя каноническое разложение чисел $a$ и $b$.
	\\
	\ensuremath{
		a = 59653 \\
		b = 6031949 \\
		\\
		\\
		a = 11^{2} \times 17 \times 29 \\
		b = 7^{2} \times 11 \times 19^{2} \times 31 \\
		\\
		gcd(a, b) = 11 \\
		lcm(a, b) =  7^{2} \times 11^2 \times 17 \times 19^{2} \times 29 \times 31
	}
	\clearpage
	%%%%%%%%%%%%%%%%%%%%%%%%%%%%%%%%%%%%%%%%%%%%%%%%%%%%%%%%%%%%%%%%%%%%%%%%%%%%%%%%%%%%%%%%%%%%%%%%%%%%%%%%%%%%%%
	\paragraph{04-1.3} Решить в целых числах линейное уравнение $ax + by = c$.
	\\
	\ensuremath{
		\begin{matrix}a = -10 & b = 25 & c = -35 \end{matrix}
		\\
		\begin{matrix}
			\text{Dividend} & \text{Divisor} & \text{Quotient} & \text{Remainder} \\
			\hline
			     & -10 &            & r_{0} = 25 \\
			\hline
			-10  &  25 &  q_{1} = 1 & r_{1} = 15 \\
			\hline
			25   &  15 &  q_{2} = 1 & r_{2} = 10 \\
			\hline
			15   &  10 &  q_{3} = 1 & r_{3} = 5 \\
			\hline
			10   &  5  &  q_{4} = 2 & r_{4} = 0
		\end{matrix} \\
		gcd(a, b) = 5 \\
		\\
		\\
		r_{0} = b \\
		r_{1} = a - ( q_{1} \cdot r_{0} ) = a - b \\
		r_{2} = r_{0} - ( q_{2} \cdot r_{1} ) = 2b - a \\
		r_{3} = r_{1} - ( q_{3} \cdot r_{2} ) = a - b - ( 2b - a ) = 2a + b \\
		\\
		\text{B\'{e}zout's identity: } gcd(a,b) = 2a + b \\
		\\
		5 \mid -35 \\
		gcd(a,b) \mid c \Rightarrow \text{ уравнение решимо в целых числах } \\
		\\
		\text{Разделим уравнение на gcd(a,b)}: \\
		-10 x + 25 y = -35 \\
		-2 x + 5 y = -7 \\
		\\
		\text{Отныне: } \\
		\begin{matrix} a = -2 & b = 5 & c = -7 & gcd(a,b) = 1 \end{matrix}
		\\
		\{x_{0}, y_{0}\} \text{ - частное решение } 
		\left \{
			\begin{array}{l}
			x = x_{0} - bt \\
			y = y_{0} + at
			\end{array}
		\right .
		\text{ - общее решение } \\
		\\
		\begin{matrix}au + bv = 1 & u = 2 & v = 1\end{matrix} \\
		a (uc) + b (vc) = c \\
		\left \{
			\begin{array}{l}
				x_{0} = uc = -14 \\
				y_{0} = -7
			\end{array}
		\right .
	}
	\clearpage
	%%%%%%%%%%%%%%%%%%%%%%%%%%%%%%%%%%%%%%%%%%%%%%%%%%%%%%%%%%%%%%%%%%%%%%%%%%%%%%%%%%%%%%%%%%%%%%%%%%%%%%%%%%%%%%
	\paragraph{04-1.4} Решить в целых числах уравнение $ax \equiv b (mod \, m)$.
	\\
	\ensuremath{
		\begin{matrix} a = 18 & b = 10 & m = 104 \end{matrix} \\
		18 x \equiv 10 (mod \, 104) \\
		\\
		\begin{matrix}
			\text{Dividend} & \text{Divisor} & \text{Quotient} & \text{Remainder} \\
			\hline
			18		&	104 	&	0 	& 	18 \\
			\hline
			104 	&	18 		&	5 	& 	14 \\
			\hline
			18 		& 	14 		& 	1 	& 	4 \\
			\hline
			14 		& 	4 		& 	3 	& 	2 \\
			\hline
			4 		& 	2 		& 	2 	& 	0
		\end{matrix} \\
		\\
		gcd(a, m) = 2 \\
		b \, \vdots \, gcd(a, m) \, \Rightarrow \, \text{ Сравнение имеет решения } \\
		\\
		\text{ Сократим сравнение до } 
			9 x \equiv 5 \, (mod \, 52) \\
		\begin{matrix} 
			a_{1} = 9 &
			b_{1} = 5 &
			m_{1} = 52
		\end{matrix} \\
		\\
		% c \equiv a_{1}^{-1} \, (mod \, 52) \\
		% c \equiv 29 \, ( mod \, 52 ) \\
		% x \equiv c a_{1} \, x 
		% 	\equiv c b_{1} 
		% 	\equiv a_{1}^{-1} b_{1} 
		% 	\equiv 29 \cdot 5 
		% 	\equiv 145 
		% 	\equiv 41 \, ( mod \, m_{1} ) \\
		m_{1} = 2^{2} \times 13 \\
		\phi(m) = \phi(13) \cdot \phi(2^{2}) = 12 \cdot 2 = 24 \\
		\\
		\overline{a_{1}^{-1}} = \overline{ a^{\phi(m) - 1} } = \overline{ 9^{23}x \, (mod \, 52) } = \overline{29}
	}
	\clearpage
	%%%%%%%%%%%%%%%%%%%%%%%%%%%%%%%%%%%%%%%%%%%%%%%%%%%%%%%%%%%%%%%%%%%%%%%%%%%%%%%%%%%%%%%%%%%%%%%%%%%%%%%%%%%%%%
	\paragraph{04-1.5} Найти число обратимых классов вычетов $\mathbb{Z}/m\mathbb{Z}$.
	\\
	\ensuremath{
		m = 969 \\
		\\
		\text{Факторизация: } 969 = 3 \times 17 \times 19 \\
		\\
		\phi(969) = \phi( 3 \cdot 17 \cdot 19 ) = \\
		= \phi(3) \cdot \phi(17) \cdot \phi(19) = 2 \cdot 16 \cdot 18 = 576 \\
		\\
		\text{Ответ: множество $\mathbb{Z}/m\mathbb{Z}$ содержит 576 обратимых вычетов}
	}
	\clearpage
	%%%%%%%%%%%%%%%%%%%%%%%%%%%%%%%%%%%%%%%%%%%%%%%%%%%%%%%%%%%%%%%%%%%%%%%%%%%%%%%%%%%%%%%%%%%%%%%%%%%%%%%%%%%%%%
	\paragraph{04-1.6} Заданы три вещественные функции: 
		\begin{itemize}
			\item $f(x) = \frac{2}{x+1}$;
			\item $g(x) = 8 x^{7} - 16$;
			\item $h(x) = cos(10 x)$.
		\end{itemize}
		\begin{enumerate}
			\item Найти композиции: \\
				\\
				\ensuremath{
					\begin{matrix}
						f \circ g \circ h & &  y = \frac{2}{ 8 (cos(10 x))^{7} - 15} \\
						\\
						h \circ f \circ g & &  y = cos( \frac{20}{8x^{7} - 15}) \\
						\\
						f \circ f \circ g & &  y = \frac{ 16x^{7} - 30 }{ 8x^{7} - 13 } \\
					\end{matrix}
				}
			\item Исследовать функции на инъективность, сюръективность, биективность на $\mathbb{R}$ \\
				\\
				\ensuremath{
					\begin{matrix}
						\hline
						\text{Функция}			& \text{Инъективность} 	& \text{Сюръективность} & \text{Биективность} 	\\
						\hline \\
						f(x) = \frac{2}{x + 1}  & \text{да} 			& \text{нет} 			& \text{нет} 			\\
						\\
						g(x) = 8 x^{7} - 16 	& \text{да} 			& \text{да} 			& \text{да} 			\\
						\\
						h(x) = cos(10x) 		& \text{нет} 			& \text{нет} 			& \text{нет} 			\\
						\\ \hline
					\end{matrix}
				}
			\item Найти обратные функции данным; если ф-ции со своими областями определения обратных не имеют, то найти обратные функции для их сужений. \\
				\ensuremath{
					\begin{matrix}
						\hline
						\text{Функция}			& \text{Обратная ф-ция} 				\\
						\hline \\
						f(x) = \frac{2}{x + 1}  &  f^{-1}(x) = \frac{2 - y}{y} 			\\
						\\
						g(x) = 8 x^{7} - 16 	&  g_{-}^{-1}(x) = \sqrt[7]{\frac{y+16}{8}} \text{  ; } x \in \mathbb{R^{*}} = \{ x | x \neq -1, \, x \in \mathbb{R} \}	\\
						\\
						h(x) = cos(10x) 		&  h_{-}^{-1}(x) = arccos(y) \text{  ;  } x \in \mathbb{R^{*}} = [-\frac{\pi}{10}; \, \frac{\pi}{10}]			\\
						\\ \hline
					\end{matrix}
				}

		\end{enumerate}

	\clearpage
	%%%%%%%%%%%%%%%%%%%%%%%%%%%%%%%%%%%%%%%%%%%%%%%%%%%%%%%%%%%%%%%%%%%%%%%%%%%%%%%%%%%%%%%%%%%%%%%%%%%%%%%%%%%%%%
	\paragraph{04-1.7} $V_{2}(Z/mZ) \longrightarrow V_{2}(Z/mZ)$, где $f(\bar{c}) = A \cdot \bar{c}$, $A \in M_{2}(Z/mZ)$. Обратима ли ф-ция $f$? В случае положительного ответа найти $f^{-1}$.
	\\
	\ensuremath{
		m = 27 \\ \\
		A = \begin{bmatrix}
			\bar{21} & \bar{19} \\
			\bar{7} & \bar{23}
		\end{bmatrix}
	}
	\clearpage
	%%%%%%%%%%%%%%%%%%%%%%%%%%%%%%%%%%%%%%%%%%%%%%%%%%%%%%%%%%%%%%%%%%%%%%%%%%%%%%%%%%%%%%%%%%%%%%%%%%%%%%%%%%%%%%
	\paragraph{04-1.8} Доказать, что множества $\mathbb{X}$ и $\mathbb{Y}$ равномощны, построив взаимно однозначное соответствие между ними.
	\\
	\ensuremath{
		\mathbb{X} = [ -28 ; 13 ) \\
		\mathbb{Y} = \mathbb{R}
		\\ \\
		y = - \frac{1}{x + 28} - \frac{1}{x - 13} \text{  ;  } x \in \mathbb{X} \text{ | К сожалению, не знаю, как сюда добавить $x = -28$}
	}
	\clearpage
	%%%%%%%%%%%%%%%%%%%%%%%%%%%%%%%%%%%%%%%%%%%%%%%%%%%%%%%%%%%%%%%%%%%%%%%%%%%%%%%%%%%%%%%%%%%%%%%%%%%%%%%%%%%%%%


\end{document}

