\documentclass[12pt]{article}
\usepackage{amsmath}
\usepackage[utf8]{inputenc}
\usepackage[russian]{babel}
\usepackage{color}
\usepackage[usenames,dvipsnames]{xcolor}
\usepackage{graphicx}
\usepackage{amsfonts}


\title{Геометрия и алгебра (часть 2). Контрольная работа №2 - Роман Гафиятуллин (192001-04)}
\author{Роман Гафиятуллин\\ БГУИР}
\begin{document}
	\begin{titlepage}
		\begin{center}
			{\Large Геометрия и алгебра (часть 2). \\ Контрольная работа №2 \\ Роман Гафиятуллин (192001-04)}
		\end{center}
	\end{titlepage}
	\clearpage
	\paragraph{04-2.1} Какую алгебраическую систему (группоид, полугруппу, моноид, группу) образует множество G относительно заданной операции? Является ли данная алгебраическая система абелевой? Множества и операции указаны ниже в соответствии с вариантами. \\
	\\
	\emph{$G$: множество всех рациональных чисел, знаменатели которых равны произведениям чисел из множества $M = \{3, 5, 7\}$, относительно операции сложения рациональных чисел.} \\
	\ensuremath{
		\text{Свойства } (G, +): \\
		\\
		\begin{matrix}
			\text{Свойство} & \text{Пример} \\
			\hline
			\text{\textbf{\emph{+}} - ассоциативен} & g_{1} + ( g_{2} + g_{3} ) = ( g_{1} + g_{2} ) + g_{3} \\
			\hline
			\text{\textbf{\emph{+}} - коммутативен} & g_{1} + g_{2} = g_{2} + g_{1} \\
			\hline
			\text{Имеет нейтральный элемент} & g + e = g; \, g, e \in G; \, e = 0 / m \\
			\hline
			\text{Не каждому $g \in G$ найдется $g^{-1}$} & g = \frac{n}{m}; \, n \notin M
		\end{matrix} \\
	}
	\\
	\\
	\textbf{$G$ является абелевым моноидом}
	\clearpage
	%%%%%%%%%%%%%%%%%%%%%%%%%%%%%%%%%%%%%%%%%%%%%%%%%%%%%%%%%%%%%%%%%%%%%%%%%%%%%%%%%%%%%%%%%%%%%%%%%%%%%%%%%%%%%%
	\paragraph{04-2.2} Разложить подстановку $f \in S_{8}$ в произведение независимых циклов и транспозиций. Определить для $f$ характер четности и порядок в группе $S_{8}$. Построить циклическую подгруппу $f$. Является ли $f$ нормальной подгруппой в $S_{8}$
 \\ \\
 	\ensuremath{
	 	f : \begin{pmatrix}
	 		1 & 2 & 3 & 4 & 5 & 6 & 7 & 8 \\
	 		3 & 6 & 4 & 5 & 7 & 8 & 1 & 2
	 	\end{pmatrix}
	 	= \begin{pmatrix} 1 & 3 & 4 & 5 & 7 \end{pmatrix} \circ
	 	  \begin{pmatrix} 2 & 6 & 8 \end{pmatrix}
	 }
	\clearpage
	%%%%%%%%%%%%%%%%%%%%%%%%%%%%%%%%%%%%%%%%%%%%%%%%%%%%%%%%%%%%%%%%%%%%%%%%%%%%%%%%%%%%%%%%%%%%%%%%%%%%%%%%%%%%%%
	\paragraph{04-2.3} Построить факторгруппу группы $(m\mathbb{Z}, +)$ по подгруппе $(n\mathbb{Z}, +)$, где $m$ и $n$ – натуральные числа, причем $m$ делит $n$, задав индуцированную операцию таблицей Кэли. Является ли факторгруппа абелевой и циклической (обосновать)? Найти все образующие элементы факторгруппы как циклической группы.
	\\ \\
	\ensuremath{ 
		m = 5; n = 25  \\
		\\
		m\mathbb{Z}/n\mathbb{Z} = \{ n\mathbb{Z}, n\mathbb{Z} + m, \, ... \, , n\mathbb{Z} + (m - 1) \} = \\
		= \{ 25\mathbb{Z}, 25\mathbb{Z} + 5, 25\mathbb{Z} + 10, 25\mathbb{Z} + 15, 25\mathbb{Z} + 20 \} \\
	}
	\clearpage
	%%%%%%%%%%%%%%%%%%%%%%%%%%%%%%%%%%%%%%%%%%%%%%%%%%%%%%%%%%%%%%%%%%%%%%%%%%%%%%%%%%%%%%%%%%%%%%%%%%%%%%%%%%%%%%
	\paragraph{04-2.4} Пусть $| <a> | = m$, $| <b>| = n$ – заданные порядки двух циклических групп. Построить все гомоморфизмы групп $f : <a> \rightarrow <b>$. Описать ядро и образ каждого гомоморфизма. Указать все мономорфизмы, эпиморфизмы и изоморфизмы, если они существуют.
	\\ \\
	\ensuremath{ m = 25; n = 15 }
	\clearpage
	%%%%%%%%%%%%%%%%%%%%%%%%%%%%%%%%%%%%%%%%%%%%%%%%%%%%%%%%%%%%%%%%%%%%%%%%%%%%%%%%%%%%%%%%%%%%%%%%%%%%%%%%%%%%%%
	\paragraph{04-2.5} Алгоритм \emph{RSA}. Задан открытый ключ $(n, e)$ и зашифрованное сообщение – число $m$. Найти секретный ключ $d$ и дешифровать сообщение – найти число $c : ( 1 \leq c \leq n - 1 )$.
	\\ \\
	\ensuremath{ n = 2257; e = 17; m = 11 }
	\clearpage
	%%%%%%%%%%%%%%%%%%%%%%%%%%%%%%%%%%%%%%%%%%%%%%%%%%%%%%%%%%%%%%%%%%%%%%%%%%%%%%%%%%%%%%%%%%%%%%%%%%%%%%%%%%%%%%
	\paragraph{04-2.6} Выяснить, является ли множество $K$ с двумя заданными на нем бинарными алгебраическими операциями кольцом, ассоциативным, коммутативным кольцом, кольцом с единицей, телом, полем. Множества и операции указаны ниже в соответствии с вариантами.
	\\ \\
	\ensuremath{
		K : \{ c \, | \, c = a + bi; \, a,b \in \mathbb{Q} \} \\
		(K, +, \cdot)
	}
	\clearpage
	%%%%%%%%%%%%%%%%%%%%%%%%%%%%%%%%%%%%%%%%%%%%%%%%%%%%%%%%%%%%%%%%%%%%%%%%%%%%%%%%%%%%%%%%%%%%%%%%%%%%%%%%%%%%%%
	\paragraph{04-2.7} Найти $gcd(f(x), g(x))$ над полем $\mathbb{Q}$ по алгоритму Евклида.
	\\ \\
	\ensuremath{
		f(x) = 3 x^5 + 8 x^4 + 7 x^3 + x^2 + 4 x \\
		g(x) = x^5 + 3 x^3 - 6 x^2 + 14x \\
	}
	\clearpage
	%%%%%%%%%%%%%%%%%%%%%%%%%%%%%%%%%%%%%%%%%%%%%%%%%%%%%%%%%%%%%%%%%%%%%%%%%%%%%%%%%%%%%%%%%%%%%%%%%%%%%%%%%%%%%%
	\paragraph{04-2.8} Для заданного $n$ найти все идеалы кольца $\mathbb{Z}/n\mathbb{Z} \, (n = 71)$, расположить их в порядке включения, указать максимальные идеалы. 
	\\ \\
	\ensuremath{}
	\clearpage
	%%%%%%%%%%%%%%%%%%%%%%%%%%%%%%%%%%%%%%%%%%%%%%%%%%%%%%%%%%%%%%%%%%%%%%%%%%%%%%%%%%%%%%%%%%%%%%%%%%%%%%%%%%%%%%
	\paragraph{04-2.9} Разложить полином $f(x) = x^3 + x^2 \in F_{2}[x]$ на неприводимые множители над полем $F_{2}$. \\
	Является ли неприводимым над полем $F_{2}$ полином $f(x)$ и максимальным идеал $(f(x))$ в кольце $F_{2}[x]$? \\
	Построить факторкольцо $F_{2}[x]/(f(x))$, задав индуцированные операции таблицами Кэли. \\
	Является ли данное факторкольцо полем? \\
	Найти характеристику $F_{2}[x]/(f(x))$.
	\\ \\
	\ensuremath{}
	\clearpage

\end{document}

