\documentclass[12pt]{article}
\usepackage{amsmath}
\usepackage[normalem]{ulem}
\usepackage[utf8]{inputenc}
\usepackage[russian]{babel}
\usepackage{color}
\usepackage[usenames,dvipsnames]{xcolor}
\usepackage{graphicx}

\title{Инструменты и Средства Программирования - Роман Гафиятуллин (192001-04)}
\author{Роман Гафиятуллин\\ БГУИР}
\begin{document}
	\begin{titlepage}
		\begin{center}
			{\Large Инструменты и Средства Программирования. \\ Контрольная работа №1 \\ Роман Гафиятуллин (192001-04)}
		\end{center}
	\end{titlepage}
	\clearpage
	\begin{enumerate}
		\item (ObjPascal) Можно ли объявить свойство в секции private? 
			\begin{description}
				\item[A] \sout{Да, если это свойство только для чтения.}
				\item[B] Да, любое свойство.
				\item[C] \sout{Да, если это простое свойство, а не свойство-массив.}
				\item[D] \sout{Нельзя.}
			\end{description}
			Почему бы и нет?

		\item (C\#) Какое утверждение верно об автосвойствах? 
			\begin{description}
				\item[A] \sout{Автосвойства должны иметь один из примитивных типов.}
				\item[B] \sout{Автосвойства могут быть свойствами только для чтения.}
				\item[C] Автосвойства могут иметь любой тип.
				\item[D] \sout{Автосвойства автоматически объявляются с модификатором public.}
			\end{description}
			Компилятор неявно создает анонимное поле для хранения значения передаваемого в авто-свойство.
			\\
			На это поле накладывается ровно тот же набор ограничений, что и на все остальные поля: никаких ограничений.

		\item (C\#) Метод содержит единственный параметр \emph{params int[] x}.
			\begin{description}
				\item[A] \sout{Этот метод можно вызвать с тремя аргументами \emph{int[]}.}
				\item[B] Этот метод можно вызвать, передав ему массив.
				\item[C] \sout{Этот метод можно вызвать с тремя аргументами \emph{object}.}
				\item[D] \sout{Этот метод можно вызвать, используя аргументы с \emph{ref} модификатором.}
			\end{description}
			Методы с переменным количеством аргументов можно вызывать как передавая аргументы один за одним, так и уложив их в массив.
			\\
			Семантически оба следующих вызова одинаковы:
			\begin{verbatim}
			MethodWithVariableArgsCount( new [] {1, 2, 3, 4} );
			MethodWithVariableArgsCount( 1, 2, 3, 4 );
			\end{verbatim}


		\item (C\#) Какая комбинация модификаторов является допустимой при объявлении поля класса?
			\begin{description}
				\item[A] \sout{\emph{readonly var.}}
				\item[B] \emph{public readonly}. 
				\item[C] \sout{\emph{public internal.}}
				\item[D] \sout{\emph{public var.}}
			\end{description}
			Всё, что с \emph{var} - бракуется сразу: поля/свойства класса/структуры необходимо объявлять явно указывая их тип.
			\\
			\emph{public} и \emph{internal} - являются аттрибутами одной категории (видимости), поэтому их совместное употребление - абсурд.
			\\
			\emph{public readonly} - допустимо.

		\item (ObjPascal) У каких классов существует \emph{RTTI}?
			\begin{description}
				\item[A] \sout{Только у тех, которые имеют \emph{public}-элементы.}
				\item[B] \sout{Только у тех, которые имеют свойства или события.}
				\item[C] \sout{Только у тех, которые имеют виртуальные методы.}
				\item[D] У всех.
			\end{description}
			RTTI доступно у всех объектов классов, унаследованных от TObject \ensuremath{ \Rightarrow } RTTI доступно у объектов всех классов.

		\item (C\#) Перечислите директивы ограничения видимости в порядке «увеличения открытости».
			\begin{description}
				\item[A] \sout{Рrivate, public, protected.}
				\item[B] \sout{Public, protected, private.}
				\item[C] Рrivate, protected, public.
				\item[D] \sout{Public, private, protected.}
			\end{description}
			\emph{private} - видно только себе; \\
			\emph{protected} - себе и потомкам; \\
			\emph{public} - всем.

		\item (ObjPascal) В классе, который наследуется непосредственно от \emph{TObject}, объявлено два поля Integer, два свойства Integer и функция. Сколько байт в динамической памяти занимает один объект класса?
			\begin{description}
				\item[A] \sout{12}
				\item[B] 8
				\item[C] \sout{20}
				\item[D] \sout{24}
			\end{description}
			На/в куче хранятся только члены-данные объекта (поля).
			\\
			Свойства являются синтаксическим сахаром, во время компиляции расскрывающимся в методы (т.н. getter и setter).
			\\
			Ответ: \begin{verbatim}2 * sizeof(Integer)\end{verbatim}
			
			Замечание: Integer -- по определению, double processor word. Да, на 32-разрядной архитектуре, два Integer, действительно займут 8 байт...

		\item (ObjPascal) Вы объявляете свойство-массив. Каким может быть тип его индекса?
			\begin{description}
				\item[A] \sout{Любым.}
				\item[B] Только числовым.
				\item[C] \sout{Любым, но не классом.}
				\item[D] \sout{Любым, но не строкой.}
			\end{description}
			Массив -- кусок памяти размером кратным размеру хранимой в нем записи.
			\\
			Доступ к элементам массива производится по адресу начального элемента плюс смещение.
			\\
			Смещение -- целое число элементов, на которое надо сместиться, с поправкой на языковые особенности Паскаля, вроде этой:
			\begin{verbatim}
			StrangeBuffer : Array[1024 .. 2048] of Byte;
			\end{verbatim}

		\item (C\#) Инициализатор объекта может задавать значения
			\begin{description}
				\item[A] \sout{Любых свойств, но не полей.}
				\item[B] Только открытых свойств.
				\item[C] \sout{Элементов (для классов-коллекций).}
				\item[D] \sout{Констант класса.}
			\end{description}
			В языке C\# есть особенность \emph{object initializer}:
			позволяют значения любым доступным полям или свойствам во время создания объекта
			

		\item (C\#) Какой модификатор нельзя использовать при описании параметра метода? 
			\begin{description}
				\item[A] \sout{\emph{params}.}
				\item[B] \emph{var}.
				\item[C] \sout{\emph{out}.}
				\item[D] \sout{\emph{ref}.}
			\end{description}
			\emph{var} инструкция компилятору, вывести (to infer) тип переменной из контекста.
			\\
			В коде всей сборки может и не быть достаточно данных о том, какого типа переменные ждет какой либо метод (к примеру \emph{public} метод в \emph{public} классе).


		\item (ObjPascal) В какой секции можно объявить конструктор класса? 
			\begin{description}
				\item[A] \sout{Только в секции protected или public.}
				\item[B] \sout{Только в секции public.}
				\item[C] В любой секции.
				\item[D] \sout{Зависит от того, где объявлен конструктор класса-предка.}
			\end{description}


		\item (C\#) Что из перечисленного является ссылочным типом?
			\begin{description}
				\item[A] \sout{bool.}
				\item[B] \sout{int.}
				\item[A] \sout{double.}
				\item[A] string.
			\end{description}

			System.String - это класс: http://msdn.microsoft.com/en-us/library/system.string.aspx

		\item (C\#)  Переменная X имеет тип short. Какой тип будет у выражения X + 10? 
			\begin{description}
				\item[A] \sout{byte.}
				\item[B] \sout{Это зависит от значения X.} 
				\item[C] int.
				\item[D] \sout{short.}
			\end{description}

			\begin{verbatim}
			X :: short
			10 :: int
			X + 10 :: int
			\end{verbatim}


		\item (C\#)  Особенностью .NET Framework является
			\begin{description}
				\item[A] \sout{Поддержка ограниченного числа языков программирования.}
				\item[B] \sout{Виртуальная среда исполнения программ.}
				\item[C] \sout{Использование сборок.}
				\item[D] \sout{Работа только в MS Windows.}
			\end{description}

			\emph{A} - количество языков теоретически не ограничено.
			\\
			\emph{B} - виртуальная машина используется не только .Net-ом, но и, к примеру - Джавой (JVM) и Эрлангом (BEAM)
			\\
			\emph{C} - 
				да, .Net использует сборки. \\
				Java - использует jar-файлы (архивы с *.class-файлами и манифестом), чем не сборка? \\
				На особенность не тянет.
			\\
			\emph{D} - .Net framework -- это обычно реализация под MS Windows. \\
				Также есть Mono Project -- довольно зрелая (в 2012 году) реализация для других платформ. Глупо от нее ожидать вещей типа COM-Interop или Windows Forms, но не завязанные на платформо-специфичные особенности приложения на ней работают: сервисы и Веб-приложения.

		\item (ObjPascal) Где размещается указатель на VMT относительно полей объекта? 
			\begin{description}
				\item[A] \sout{В VMT родительского класса.}
				\item[B] \sout{В динамической памяти, после полей объекта.}
				\item[C] \sout{В динамической памяти, перед полями объекта.}
				\item[D] В стеке или сегменте данных, перед полями объекта.
			\end{description}

		\item (C\#)  Какие модификаторы можно использовать при объявлении локальной переменной метода? 
			\begin{description}
				\item[A] \sout{Модификатор private или readonly.}
				\item[B] \sout{Модификатор readonly.}
				\item[C] \sout{Модификатор static или readonly.}
				\item[D] Никаких модификаторов использовать нельзя.
			\end{description}

		\item (ObjPascal)  Что содержит переменная типа «указатель на метод» и какой её размер?
			\begin{description}
				\item[A] \sout{Указатель на VMT и значение self; 8 байт.}
				\item[B] \sout{Адрес метода и параметры метода; размер зависит от количества параметров.}
				\item[C] \sout{Адрес метода; 4 байта.}
				\item[D] Адрес метода и значение self; 8 байт.
			\end{description}

		\item (C\#) Можно ли обнулить массив int[], перебирая его элементы в foreach и присваивая им 0? 
			\begin{description}
				\item[A] \sout{Да.} 
				\item[B] \sout{Да, но только если массив не readonly.}
				\item[C] Нет.
			\end{description}

			Нет - это \"foreach iteration variable\", её изменять нельзя.
			\\
			Логично, многие коллекции меняют свою структуру, при изменении их элементов. Итератор этого может не перенести.

		\item (C\#) Какая комбинация модификаторов является допустимой при объявлении индексатора? 
			\begin{description}
				\item[A] \sout{public var.}
				\item[B] \sout{static.}
				\item[C] public.
				\item[D] \sout{public internal.}
			\end{description}

		\item (C\#) Переменную какого типа можно использовать в качестве селектора в операторе switch-case? 
			\begin{description}
				\item[A] \sout{int[].}
				\item[B] \sout{Переменную пользовательского класса.}
				\item[C] bool.
				\item[D] \sout{double.}
			\end{description}


	\end{enumerate}
\end{document}