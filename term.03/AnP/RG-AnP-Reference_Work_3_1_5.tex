\documentclass[12pt]{article}
\usepackage{amsmath}
\usepackage[utf8]{inputenc}
\usepackage[russian]{babel}
\usepackage{color}
\usepackage[usenames,dvipsnames]{xcolor}
\usepackage{graphicx}
\usepackage{listings}

\usepackage{color}

\definecolor{dkgreen}{rgb}{0,0.6,0}
\definecolor{gray}{rgb}{0.5,0.5,0.5}
\definecolor{mauve}{rgb}{0.58,0,0.82}

\lstset{frame=tb,
  % language=C,
  aboveskip=3mm,
  belowskip=3mm,
  showstringspaces=false,
  columns=flexible,
  basicstyle={\small\ttfamily},
  numbers=left,
  numberstyle=\tiny\color{gray},
  keywordstyle=\color{blue},
  commentstyle=\color{dkgreen},
  stringstyle=\color{mauve},
  breaklines=true,
  breakatwhitespace=true
  tabsize=2
}

\title{Алгоритмизация и Программирование. Семестр №3. Индивидуальная работа №1 - Роман Гафиятуллин (192001-04)}
\author{Роман Гафиятуллин\\ БГУИР}
\begin{document}
	\begin{titlepage}
		\begin{center}
			{\Large Алгоритмизация и Программирование. \\ Семестр №3. Индивидуальная работа №1 \\ Роман Гафиятуллин (192001-04)}
		\end{center}
	\end{titlepage}
	%%%%%%%%%%%%%%%%%%%%%%%%%%%%%%%%%%%%%%%%%%%%%%%%%%%%%%%%%%%%%%%%%%%%%%%%%%%%%%%%%%%%%%%%%%%%%%%%%%%%%%%%%%%%%%%%%
	\clearpage
	\paragraph{1} Цена на молоко. Фермер в начале каждой зимы повышает отпускную цену на молоко на $d\%$, а каждым летом – снижает на столько же процентов. 
					Изменится ли цена на молоко и если да, то в какую сторону и на сколько через $n$ лет?
	\\
	\\
	\emph{Решение:}
	\\
	Цена на молоко через $n$ вычисляется согласно следующему закону ( берутся в расчет только летние цены ):
	\\
	\ensuremath{
		\\
		\hbox{Рекурентно [ за время равное $O(n)$ и память равную $O(n)$ ] } \\
		p_{n} = p_{n - 1} \cdot (1 + d/100) \cdot (1 - d/100) \\
		\\
		\hbox{Используя формулу геометрической прогрессии [ за время - $O(log_{2} n)$ и память $O(1)$ ]} \\
		p_{n} = p_{0} \cdot ( (1 + d/100) \cdot (1 - d/100) )^{n} \\
	}
	\\
	Зимняя цена следующая за $p_{n}$: $p_{n} = p_{n} * (1 + d/100)$
	\\
	\\
	Коментарий: несмотря на то, что существует рекурсивное решение требующее память равную $O(1)$ (сведение решения к хвостовой рекурсии либо выпрямление рекурсии в цикл),
	следует помнить, что для многих рекурентных формул существуют более выгодные аналитические решения.
	\clearpage
	\paragraph{2} Найти все натуральные числа от 1 до 1000, которые совпадают с последними разрядами своих квадратов, например: $25^2 = 625$; $76^2 = 5676$.\\
	\\
	\emph{Решение:}
	Функция определения количества цифр требуемых для записи числа в системе счисления с базисом $r$ (en: r - radix, dc - digits count). \\ \\
	\ensuremath{
		dc_{radix} (n) = ceil( log_{radix}( n + 1 ) ) \\
		\hbox{ где } ceil( f ) \hbox{ - функция округление до целого в большую сторону} \\
	}
	\\
	Функция получения числа образуемого двумя младшими разрядами числа (en: lsd - least significant digits ). \\ \\
	\ensuremath{
		lsd_{radix} (n) = n \, mod \, (radix)^{dc_{radix}(n)}
	}
	\\
	Функция проверки числа на описанное в задаче условие: \\ \\
	\ensuremath{
		test_{radix} (n) = eq( n\, , \, l2sd_{radix}( n^{2} ) )
	} \\
	где $eq( n_{1} \, , \, n_{2} )$ – функция булевого типа равная истине только при равных друг другу аргументах
	\\
	\clearpage
	\paragraph{3} Нахождение НОД при помощи алгоритма Евклида. \\
	\\
	\emph{Решение:}
	\\
	Довольно тривиально и не требует пояснения.
	\clearpage
	\lstinputlisting[language=C]{rg-anp-rw-3-1-5.c}
	\clearpage
	\lstinputlisting[]{rg-anp-rw-3-1-5.output}
\end{document}